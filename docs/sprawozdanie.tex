% Options for packages loaded elsewhere
\PassOptionsToPackage{unicode}{hyperref}
\PassOptionsToPackage{hyphens}{url}
\documentclass[
]{article}
\usepackage{graphicx}
\usepackage{float}
\usepackage{xcolor}
\usepackage{amsmath,amssymb}
\setcounter{secnumdepth}{-\maxdimen} % remove section numbering
\usepackage{iftex}
\ifPDFTeX
  \usepackage[T1]{fontenc}
  \usepackage[utf8]{inputenc}
  \usepackage{textcomp} % provide euro and other symbols
\else % if luatex or xetex
  \usepackage{unicode-math} % this also loads fontspec
  \defaultfontfeatures{Scale=MatchLowercase}
  \defaultfontfeatures[\rmfamily]{Ligatures=TeX,Scale=1}
\fi
\usepackage{lmodern}
\ifPDFTeX\else
  % xetex/luatex font selection
\fi
% Use upquote if available, for straight quotes in verbatim environments
\IfFileExists{upquote.sty}{\usepackage{upquote}}{}
\IfFileExists{microtype.sty}{% use microtype if available
  \usepackage[]{microtype}
  \UseMicrotypeSet[protrusion]{basicmath} % disable protrusion for tt fonts
}{}
\makeatletter
\@ifundefined{KOMAClassName}{% if non-KOMA class
  \IfFileExists{parskip.sty}{%
    \usepackage{parskip}
  }{% else
    \setlength{\parindent}{0pt}
    \setlength{\parskip}{6pt plus 2pt minus 1pt}}
}{% if KOMA class
  \KOMAoptions{parskip=half}}
\makeatother
\usepackage{longtable,booktabs,array}
\usepackage[top=3cm, bottom=3cm, left=2cm, right=2cm]{geometry}
\usepackage{calc} % for calculating minipage widths
% Correct order of tables after \paragraph or \subparagraph
\usepackage{etoolbox}
\makeatletter
\patchcmd\longtable{\par}{\if@noskipsec\mbox{}\fi\par}{}{}
\makeatother
% Allow footnotes in longtable head/foot
\IfFileExists{footnotehyper.sty}{\usepackage{footnotehyper}}{\usepackage{footnote}}
\makesavenoteenv{longtable}
\usepackage{graphicx}
\makeatletter
\newsavebox\pandoc@box
\newcommand*\pandocbounded[1]{% scales image to fit in text height/width
  \sbox\pandoc@box{#1}%
  \Gscale@div\@tempa{\textheight}{\dimexpr\ht\pandoc@box+\dp\pandoc@box\relax}%
  \Gscale@div\@tempb{\linewidth}{\wd\pandoc@box}%
  \ifdim\@tempb\p@<\@tempa\p@\let\@tempa\@tempb\fi% select the smaller of both
  \ifdim\@tempa\p@<\p@\scalebox{\@tempa}{\usebox\pandoc@box}%
  \else\usebox{\pandoc@box}%
  \fi%
}
% Set default figure placement to htbp
\def\fps@figure{htbp}
\makeatother
\setlength{\emergencystretch}{3em} % prevent overfull lines
\providecommand{\tightlist}{%
  \setlength{\itemsep}{0pt}\setlength{\parskip}{0pt}}
\usepackage{bookmark}
\IfFileExists{xurl.sty}{\usepackage{xurl}}{} % add URL line breaks if available
\urlstyle{same}
\hypersetup{
  pdftitle={Dokumentacja wstępna: Analiza długich rezerwacji (zadanie 7)},
  pdfauthor={Damian D'Souza; Kamil Marszałek},
  hidelinks,
  pdfcreator={LaTeX via pandoc}}

\title{Dokumentacja wstępna: Analiza długich rezerwacji (zadanie 7)}
\author{Damian D'Souza \and Kamil Marszałek}
\date{2025-12-20}

\begin{document}
\maketitle

\subsection{1. Definicja problemu}\label{definicja-problemu}

Obecnie brakuje nam wiedzy o czynnikach, które wpływają na długość
rezerwacji. Nie wiemy, dlaczego niektórzy klienci rezerwują noclegi na
dłuższe okresy (np. 7 dni+).

To utrudnia pracę konsultantom - nie mogą skutecznie doradzać
właścicielom, jakie udogodnienia warto dodać, żeby przyciągnąć klientów
szukających dłuższych pobytów.

\textbf{Cel projektu:} Zidentyfikować kluczowe czynniki (cechy oferty i
profilu klienta), które zwiększają prawdopodobieństwo długiej
rezerwacji. Dzięki temu konsultanci będą mogli konkretnie wskazać
oferentom, co zmienić w ofercie (np. dodać szybkie Wi-Fi, wprowadzić
rabaty tygodniowe), żeby zwiększyć szansę na dłuższe rezerwacje.

\subsection{2. Podejście modelowe}\label{podejux15bcie-modelowe}

\subsubsection{2.1 Co modelujemy}\label{co-modelujemy}

\begin{itemize}
	\item
	      Zmienna wyjściowa \emph{y}:

	      \begin{itemize}
		      \tightlist
		      \item
		            \texttt{1} -- rezerwacja „Długa'',\\
		      \item
		            \texttt{0} -- rezerwacja „Krótka''.
	      \end{itemize}
	\item
	      Robocza definicja długiej rezerwacji: pobyt np. powyżej 7 dni
	\item
	      Rodzaj zadania: klasyfikacja binarna.
\end{itemize}

\subsubsection{2.2 Wybór modelu}\label{wybuxf3r-modelu}

Potrzebujemy modelu interpretowalnego, który pozwoli zrozumieć, które
cechy są najważniejsze.

Rozważane algorytmy:

\begin{itemize}
	\tightlist
	\item
	      Regresja logistyczna
	\item
	      Drzewa decyzyjne (Gradient Boosting)
\end{itemize}

Oba umożliwiają łatwe wyodrębnienie wag cech.

\subsubsection{2.3 Kryteria sukcesu}\label{kryteria-sukcesu}

\subsubsection{2.3.1 Kryterium biznesowe}\label{kryterium-biznesowe}

\textbf{Cel biznesowy:} Dostarczenie konsultantom konkretnych,
wdrożeniowych rekomendacji dla właścicieli nieruchomości co pozwoli
zwiększyć liczbę długich rezerwacji o 15\%.

\textbf{Mierniki sukcesu biznesowego:}

\begin{itemize}
	\tightlist
	\item
	      Model identyfikuje minimum 3-5 kluczowych cech, które właściciele mogą
	      zmodyfikować (np. dodanie Wi-Fi, wyposażonej kuchni)
	\item
	      Rekomendacje są zrozumiałe dla konsultantów bez technicznego
	      background'u
\end{itemize}

\subsubsection{2.3.2 Kryterium analityczne}\label{kryterium-analityczne}

\textbf{Metryki:} Ze względu na potencjalnie niezbalansowane dane
(długie rezerwacje mogą być rzadsze), używamy metryk odpornych na
dysproporcje klas:

\begin{itemize}
	\tightlist
	\item
	      \textbf{ROC AUC} - główna metryka
	\item
	      \textbf{PR AUC} - metryka uzupełniająca, szczególnie ważna przy silnie
	      niezbalansowanych danych, gdzie ROC AUC może dawać fałszywie
	      optymistyczne wyniki
\end{itemize}

\textbf{Model bazowy do porównania:} (Opracowane modele zostały opisane
niżej po analizie danych)

\textbf{Model większościowy} - zawsze przewiduje klasę większościową
(najprawdopodobniej ``krótka rezerwacja''). Taki model osiąga ROC AUC
ok. 0.5 (równoważne losowemu zgadywaniu).

\textbf{Cel analityczny:}

Model musi spełniać następujące kryteria:

\begin{itemize}
	\tightlist
	\item
	      \textbf{Przewidywalność:} Wykazać, że długość rezerwacji jest w ogóle
	      przewidywalna na podstawie dostępnych danych
	\item
	      \textbf{Praktyczna użyteczność:} Osiągnąć \textbf{ROC AUC
		      \textgreater{} 0.7}, co oznacza wyraźną poprawę względem modelu
	      bazowego (0.5) i wskazuje na realną wartość predykcyjną
	\item
	      \textbf{Interpretowalność:} Umożliwić identyfikację najważniejszych
	      cech z jasną interpretacją ich wpływu
\end{itemize}

\textbf{Proces ustalania progów:}

\begin{enumerate}
	\def\labelenumi{\arabic{enumi}.}
	\tightlist
	\item
	      \textbf{Analiza danych:}

	      \begin{itemize}
		      \tightlist
		      \item
		            Zbadanie rozkładu klas (\% długich vs.~krótkich rezerwacji)
		      \item
		            Ocena jakości i kompletności danych
	      \end{itemize}
	\item
	      \textbf{Wstępne modelowanie:}

	      \begin{itemize}
		      \tightlist
		      \item
		            Zbudowanie prostego modelu
		      \item
		            Ustalenie baseline'u dla ROC AUC
		      \item
		            Ocena czy problem jest w ogóle przewidywalny
	      \end{itemize}
	\item
	      \textbf{Ustalenie progów:}

	      \begin{itemize}
		      \tightlist
		      \item
		            Na podstawie baseline'u określenie realistycznego celu
		      \item
		            Walidacja czy osiągnięta skuteczność przekłada się na wartość
		            biznesową (czy zidentyfikowane czynniki są modyfikowalne przez
		            właścicieli)
	      \end{itemize}
\end{enumerate}

\subsection{3. Analiza danych}\label{analiza-danych}

\subsubsection{3.1 Dostępne atrybuty}\label{dostux119pne-atrybuty}

W tej sekcji omówimy różne atrybuty dostępne w naszym zestawie danych
oraz ich znaczenie.

\paragraph{Dane listings.csv}\label{dane-listings.csv}
\mbox{}

W pliku \texttt{listings.csv} znajduje się 79 atrybutów między innymi:

\begin{itemize}
	\tightlist
	\item
	      \texttt{id}: Unikalny identyfikator oferty.
	\item
	      \texttt{name}: Nazwa oferty.
	\item
	      \texttt{price}: Cena oferty.
	\item
	      \texttt{room\_type}: Typ pokoju (np. całe mieszkanie, prywatny pokój).
	\item
	      \texttt{amenities}: Lista udogodnień dostępnych w ofercie np. czy jest
	      pralka.
	\item
	      tekstowe pola opisowe np. \texttt{description},
	      \texttt{neighborhood\_overview}.
\end{itemize}

\paragraph{Dane reviews.csv}\label{dane-reviews.csv}
\mbox{}

W pliku \texttt{reviews.csv} znajduje się 6 atrybutów:

\begin{itemize}
	\tightlist
	\item
	      \texttt{listing\_id}: Unikalny identyfikator oferty, do której odnosi
	      się recenzja.
	\item
	      \texttt{id}: Unikalny identyfikator recenzji.
	\item
	      \texttt{date}: Data dodania recenzji.
	\item
	      \texttt{reviewer\_id}: Unikalny identyfikator recenzenta.
	\item
	      \texttt{reviewer\_name}: Imię recenzenta.
	\item
	      \texttt{comments}: Treść recenzji.
\end{itemize}

\paragraph{Dane sessions.csv}\label{dane-sessions.csv}
\mbox{}

W pliku \texttt{sessions.csv} znajduje się 7 atrybutów:

\begin{itemize}
	\tightlist
	\item
	      \texttt{action}: Typ akcji wykonanej przez użytkownika (np.
	      wyszukiwanie, obejrzenie oferty, rezerwacja oferty, odwołanie
	      rezerwacji).
	\item
	      \texttt{user\_id}: Identyfikator użytkownika wykonującego akcję.
	\item
	      \texttt{timestamp}: Znacznik czasu wykonania akcji.
	\item
	      \texttt{listing\_id}: Identyfikator oferty, na której wykonano.
	\item
	      \texttt{booking\_date}: Data rezerwacji oferty.
	\item
	      \texttt{booking\_duration}: (Prawdopodobnie) data zakończenia
	      rezerwacji, chociaż nazwa sugeruje że to powinna być liczba!.
	\item
	      \texttt{booking\_id}: Identyfikator rezerwacji oferty.
\end{itemize}

\paragraph{Dane users.csv}\label{dane-users.csv}
\mbox{}

W pliku \texttt{users.csv} znajduje się 7 atrybutów:

\begin{itemize}
	\tightlist
	\item
	      \texttt{id}: Identyfikator użytkownika.
	\item
	      \texttt{name}: Imię użytkownika.
	\item
	      \texttt{surname}: Nazwisko użytkownika.
	\item
	      Dane adresowe (prawdopodobnie adres zamieszkania użytkownika):

	      \begin{itemize}
		      \tightlist
		      \item
		            \texttt{city}: Miasto.
		      \item
		            \texttt{street}: Ulica.
		      \item
		            \texttt{street\_number}: Numer domu.
		      \item
		            \texttt{postal\_code}: Kod pocztowy.
	      \end{itemize}
\end{itemize}

\subsubsection{3.2 Co jest istotne dla naszego
	problemu?}\label{co-jest-istotne-dla-naszego-problemu}

Chcemy dowiedzieć się, na podstawie jakich czynników użytkownicy
rezerwują długie pobyty. Do zamodelowania problemu niezbędne będą dane
występujące w pliku \texttt{sessions.csv}, gdyż to właśnie stamtąd
możemy uzyskać informacje o długości rezerwacji. Data rozpoczęcia pobytu
i data zakończenia pobytu (lub czas trwania rezerwacji - być może jest
błąd w kolumnie) są kluczowymi atrybutami do analizy. Stworzymy na ich
podstawie pojęcie binarne - \texttt{long\_stay} - i właśnie to chcemy,
żeby model przewidywał jak najlepiej po wytrenowaniu. Z kolei z pliku
\texttt{listings.csv} możemy pozyskać informacje o cechach ofert, które
mogą wpływać na decyzje użytkowników dotyczące długości pobytu, takie
jak cena, typ pokoju czy dostępne udogodnienia. Plik \texttt{users.csv}
może dostarczyć dodatkowych informacji demograficznych o użytkownikach,
które również mogą mieć wpływ na ich wybory rezerwacyjne. Plik
\texttt{reviews.csv} może być mniej istotny dla naszego konkretnego
problemu, ale może dostarczyć dodatkowych informacji o jakości ofert i
doświadczeniach użytkowników.

Atrybuty w pliku \texttt{listings.csv} można podzielić zgrubsza na kilka
kategorii:

\begin{itemize}
	\tightlist
	\item
	      Ekonomia pobytu:

	      \begin{itemize}
		      \tightlist
		      \item
		            \texttt{price}: Cena prawdopodobnie za noc. Jest ona podana z walutą
		            (np. ``\$100.00''). Należałoby ją przekształcić na wartość
		            numeryczną.
		      \item
		            \texttt{minimum\_nights}: Minimalna liczba nocy, na jaką można
		            dokonać rezerwacji. Niektóre lokale są dostępne np. tylko na dłuższe
		            pobyty (\textgreater30 dni).
		      \item
		            \texttt{maximum\_nights}: Maksymalna liczba nocy, na jaką można
		            dokonać rezerwacji. Niektóre obiekty mogą zezwalać tylko na krótkie
		            pobyty np. ich maksymalna liczba nocy to 7.
		      \item
		            \texttt{has\_availability}: Czy lokal jest dostępny do rezerwacji.
	      \end{itemize}
	\item
	      Standard i udogodnienia:

	      \begin{itemize}
		      \tightlist
		      \item
		            \texttt{room\_type}: Czy to całe mieszkanie, czy pokój prywatny lub
		            współdzielony. Przy długich pobytach potencjalnie większa przestrzeń
		            i prywatność mogą być ważne.
		      \item
		            \texttt{amenities}: Zawiera listę udogodnień oferowanych przez
		            lokal, takich jak pralka, kuchnia, Wi-Fi, akceptowanie zwierząt itp.
		            Udogodnienia te mogą znacząco wpłynąć na komfort długiego pobytu.
		            Trzeba będzie przetworzyć tę kolumnę, aby wyodrębnić konkretne
		            udogodnienia jako cechy binarne (np. \texttt{has\_washing\_machine},
		            \texttt{has\_kitchen}).
		      \item
		            \texttt{accommodates}: Liczba osób, które lokal może pomieścić. Dla
		            dłuższych pobytów może być istotne, czy lokal jest odpowiedni dla
		            większych grup lub rodzin.
		      \item
		            \texttt{bedrooms}, \texttt{beds}: Liczba sypialni i łóżek w lokalu.
		            Więcej sypialni i łóżek może być korzystne dla dłuższych pobytów,
		            zwłaszcza dla rodzin lub grup.
		      \item
		            \texttt{bathrooms}: Liczba łazienek w lokalu. Więcej łazienek może
		            zwiększyć komfort podczas dłuższych pobytów.
		      \item
		            \texttt{bathrooms\_text}: Opis łazienek - czy są wspólne, prywatne
	      \end{itemize}
	\item
	      Lokalizacja i otoczenie:

	      \begin{itemize}
		      \tightlist
		      \item
		            \texttt{neighbourhood\_cleansed}: Nazwa dzielnicy, w której znajduje
		            się lokal.
		      \item
		            \texttt{neighbourhood\_overview}: Opis okolicy. Dla długich pobytów
		            ważne może być, czy okolica jest spokojna, bezpieczna i czy oferuje
		            udogodnienia takie jak sklepy, restauracje itp. W przeciwieństwie do
		            krótkich pobytów, gdzie lokalizacja w centrum miasta, blisko miejsc
		            turystycznie atrakcyjnych może być bardziej pożądana, długie pobyty
		            mogą wymagać bardziej zrównoważonej lokalizacji.
		      \item
		            \texttt{latitude}, \texttt{longitude}: Dokładne współrzędne
		            geograficzne lokalu. Mogą być użyteczne do analizy lokalizacji i jej
		            wpływu na decyzje rezerwacyjne.
	      \end{itemize}
	\item
	      Zaufanie i wiarygodność:

	      \begin{itemize}
		      \tightlist
		      \item
		            \texttt{host\_is\_superhost}: Czy gospodarz ma status superhosta.
		            Gospodarze z tym statusem mogą być postrzegani jako bardziej
		            wiarygodni, co może wpływać na decyzje dotyczące długich pobytów.
		      \item
		            \texttt{host\_response\_time}: Czas odpowiedzi gospodarza. Szybka
		            odpowiedź może być ważna dla gości planujących dłuższe pobyty.
		            Lepiej, gdy gospodarz szybko odpowiada na zapytania np. within an
		            hour.
		      \item
		            \texttt{review\_scores\_rating}: Ogólna ocena lokalu.
		      \item
		            \texttt{review\_scores\_value}: Ocena stosunku jakości do ceny.
		      \item
		            \texttt{number\_of\_reviews}: Liczba recenzji. Większa liczba
		            recenzji może świadczyć o popularności i zaufaniu do lokalu.
	      \end{itemize}
\end{itemize}

Z pliku \texttt{users.csv} nie użyjemy danych personalnych takich jak
imię, nazwisko czy ulica. Być może użyjemy atrybut \texttt{city}, aby
zobaczyć, czy lokalizacja użytkownika ma wpływ na długość rezerwacji np.
czy osoby z większych miast rezerwują dłuższe pobyty.

Dane, które są w pliku \texttt{reviews.csv} prawdopodobnie nie będą
użyte w naszym modelu, ponieważ recenzje są dodawane po zakończeniu
pobytu i nie wpływają na decyzję o długości rezerwacji. Ważniejsza może
być zagregowana ocena, które jest już w pliku \texttt{listings.csv}.

\subsubsection{3.3 Problemy ze złączeniami danych między
	plikami}\label{problemy-ze-zux142ux105czeniami-danych-miux119dzy-plikami}

Aby wykorzystać dane z różnych plików, musimy je łączyć po kluczach.

\paragraph{\texorpdfstring{Złączenie \texttt{sessions.csv}
		\(\leftrightarrow\)
		\texttt{listings.csv}}{Złączenie sessions.csv i listings.csv}}\label{zux142ux105czenie-sessions.csv-leftrightarrow-listings.csv}
\mbox{}

Dla \texttt{listings.csv} i \texttt{sessions.csv} kluczami są
odpowiednio \texttt{listings.id} oraz \texttt{sessions.listing\_id}.
Analizę pokrycia wykonujemy w dwóch ujęciach: pokrycie wierszy po
złączeniu oraz pokrycie unikalnych kluczy (różnorodności ofert).

\begin{itemize}
	\item
	      Bez filtrowania po \texttt{action}: ok. 74\% rekordów
	      \texttt{sessions.csv} ma niepusty \texttt{listing\_id}, a spośród nich
	      ok. 67.6\% znajduje dopasowanie w \texttt{listings.csv} (łączny row
	      coverage \textasciitilde50.1\%). Jednocześnie tylko ok. 13.2\%
	      unikalnych \texttt{listing\_id} obserwowanych w \texttt{sessions.csv}
	      występuje w \texttt{listings.csv}, co sugeruje, że
	      \texttt{listings.csv} jest ograniczonym podzbiorem ofert, a
	      dopasowania dotyczą głównie najczęściej występujących listingów.
	\item
	      Dla akcji \texttt{book\_listing} (rezerwacje): ok. 80\% rekordów ma
	      niepusty \texttt{listing\_id}, jednak tylko ok. 7.7\% z nich znajduje
	      dopasowanie w \texttt{listings.csv} (łączny row coverage
	      \textasciitilde6.2\%). Oznacza to, że cechy ofert z
	      \texttt{listings.csv} będą dostępne tylko dla niewielkiej części
	      rezerwacji; trening modelu oparty głównie o cechy ofert może być przez
	      to niemożliwy lub silnie obciążony selekcją danych.
\end{itemize}

Jednak zauważyliśmy, że wiele brakujących \texttt{id} w
\texttt{listings.csv} można by było odzyskać, analizując kolumnę
\texttt{listing\_url} i wyciągając z niej identyfikatory ofert. Po
naprawie braków w \texttt{listings.csv}, ponowiliśmy analizę złączeń:

\begin{itemize}
	\tightlist
	\item
	      Bez filtrowania po \texttt{action}: pokrycie wierszy po złączeniu
	      wzrosło do ok. 60.3\%, a pokrycie unikalnych kluczy do ok. 15.8\%.
	      Natomiast dopasowanie wzrosło do ok. 81.4\% rekordów z niepustym
	      \texttt{listing\_id}.
	\item
	      Dla akcji \texttt{book\_listing}: pokrycie wierszy po złączeniu
	      wzrosło do ok. 7.4\%, a pokrycie unikalnych kluczy do ok. 9.4\%.
	      Dopasowanie wzrosło do ok. 9.3\% rekordów z niepustym
	      \texttt{listing\_id}. Mimo poprawy, cechy ofert z \texttt{listings}
	      nadal będą dostępne tylko dla małej części rezerwacji, więc będzie
	      problem z trenowaniem.
\end{itemize}

Następnie poczyniliśmy obserwację, że wiele rekordów w
\texttt{sessions.csv} ma nieustawione atrybut \texttt{action}, mimo że
posiadają daty rezerwacji i wyglądają jak rezerwacje. Dzięki temu
pozyskaliśmy dodatkowe 15000 rekordów, które są rezerwacjami. Po
ponownym przeanalizowaniu złączeń po naprawie \texttt{listings.csv} i
uzupełnieniu \texttt{sessions.csv}:

\begin{itemize}
	\tightlist
	\item
	      Bez filtrowania po \texttt{action}: nie zmieniły się wyniki złączenia.
	\item
	      Dla akcji \texttt{book\_listing} (rezerwacje): pokrycie wierszy po
	      złączeniu wyniosło ok. 7.4\%, a pokrycie unikalnych kluczy to ok.
	      9.5\% (minimalny wzrost). Dopasowanie wyniosło ok. 9.2\% rekordów z
	      niepustym \texttt{listing\_id} (minimalny spadek). Dalej cechy ofert z
	      \texttt{listings} będą dostępne tylko dla małej części rezerwacji.
	      Niemniej jednak dodatkowe rekordy rezerwacji mogą pomóc w trenowaniu
	      modelu.
\end{itemize}

\paragraph{\texorpdfstring{Złączenie \texttt{sessions.csv}
		\(\leftrightarrow\)
		\texttt{users.csv}}{Złączenie sessions.csv i users.csv}}\label{zux142ux105czenie-sessions.csv-leftrightarrow-users.csv}
\mbox{}

Klucz łączący to \texttt{sessions.user\_id} oraz \texttt{users.id}.

\begin{itemize}
	\item
	      Bez filtrowania po \texttt{action}: ok. 80\% rekordów
	      \texttt{sessions.csv} ma niepusty \texttt{user\_id}, a spośród nich
	      ok. 80\% znajduje dopasowanie w \texttt{users.csv} (łączny row
	      coverage \textasciitilde64\%). Pokrycie unikalnych \texttt{user\_id}
	      wynosi ok. 80\%, co sugeruje, że dla większości użytkowników obecnych
	      w sesjach istnieją odpowiadające im dane w \texttt{users.csv}, choć
	      ok. 20\% identyfikatorów nie jest możliwe do sparowania.
	\item
	      Dla akcji \texttt{book\_listing} (rezerwacje): wyniki są bardzo
	      zbliżone.
\end{itemize}

\paragraph{\texorpdfstring{Złączenie \texttt{reviews.csv}
		\(\leftrightarrow\)
		\texttt{users.csv}}{Złączenie reviews.csv i users.csv}}\label{zux142ux105czenie-reviews.csv-leftrightarrow-users.csv}
\mbox{}

Klucz łączący to \texttt{reviews.reviewer\_id} oraz \texttt{users.id}.

\begin{itemize}
	\tightlist
	\item
	      Ok. 80\% rekordów \texttt{reviews.csv} ma niepusty
	      \texttt{reviewer\_id}, a spośród nich ok. 80\% znajduje dopasowanie w
	      \texttt{users.csv} (łączny row coverage \textasciitilde64\%). Pokrycie
	      unikalnych \texttt{reviewer\_id} wynosi ok. 80\%.
\end{itemize}

\paragraph{\texorpdfstring{Złączenie \texttt{reviews.csv}
		\(\leftrightarrow\)
		\texttt{listings.csv}}{Złączenie reviews.csv i listings.csv}}\label{zux142ux105czenie-reviews.csv-leftrightarrow-listings.csv}
\mbox{}

Klucz łączący to \texttt{reviews.listing\_id} oraz \texttt{listings.id}.

\begin{itemize}
	\tightlist
	\item
	      Ok. 80.1\% rekordów w \texttt{reviews.csv} ma niepusty
	      \texttt{listing\_id}. Jednak tylko ok. 7.6\% z nich znajduje
	      dopasowanie w \texttt{listings.csv}, co daje łączne pokrycie wierszy
	      po złączeniu na poziomie ok. 6.1\%. Pokrycie po unikalnych kluczach
	      jest również niskie: tylko ok. 8.0\% unikalnych \texttt{listing\_id} z
	      \texttt{reviews.csv} występuje w \texttt{listings.csv}
	      (\texttt{unique\_key\_coverage} \textasciitilde0.0803). Wskazuje to,
	      że \texttt{listings.csv} jest ograniczonym podzbiorem ofert i nie
	      pozwala na wzbogacenie większości recenzji o cechy oferty.
\end{itemize}

\subsubsection{3.4 Braki danych}\label{braki-danych}

Przeanalizowaliśmy braki danych w plikach. Braki są dosyć duże:

\begin{itemize}
	\tightlist
	\item
	      Plik \texttt{listings.csv} ma braki we wszystkich kolumnach - minimum
	      18\% w kolumnie \texttt{host\_has\_profile\_pic}, maksimum 100\% w
	      kolumnie \texttt{calendar\_updated}. Jednak dla większości kolumn
	      braki są w przedziale od ok. 20\% do 40\%. Po naprawie kolumny
	      \texttt{id} (opisanej wyżej) braki w tej kolumnie zostały zredukowane
	      do ok. 4.6\%.
	\item
	      Plik \texttt{sessions.csv} również ma braki we wszystkich kolumnach
	      minimum ok. 20\% w kolumnach \texttt{action}, \texttt{user\_id},
	      \texttt{timestamp}, maksymalne braki to ok. 94\% w kolumnach
	      \texttt{booking\_date}, \texttt{booking\_duration},
	      \texttt{booking\_id}. Po zastosowaniu naprawy i uzupełnienia
	      \texttt{action} o brakujące wartości, braki w kolumnie \texttt{action}
	      zmalały do 19\%. Braki w kolumnach związanych z rezerwacjami są
	      zrozumiałe, ponieważ nie każda sesja kończy się rezerwacją. Jednak
	      zmienną celową \texttt{long\_stay} można będzie utworzyć tylko dla
	      około 4.7\% rekordów, gdyż braki w kolumnach \texttt{booking\_date} i
	      \texttt{booking\_duration} nie występują jednocześnie.
	\item
	      Pliki \texttt{users.csv} i \texttt{reviews.csv} mają braki we
	      wszystkich kolumnach - minimum na poziomie około 20\%.
\end{itemize}

\subsubsection{3.5 Definicja targetu}\label{definicja-targetu}

Na potrzeby naszego problemu zbudujemy dataset
\texttt{reservations.csv}, który będzie zawierał tylko rekordy z
\texttt{sessions.csv}, które reprezentują rezerwacje (czyli mają
uzupełnione kolumny \texttt{booking\_date} i
\texttt{booking\_duration}). Kolumnę \texttt{booking\_date} nazwiemy
\texttt{checkin}, a kolumnę \texttt{booking\_duration} nazwiemy
\texttt{checkout}. Następnie na ich podstawie zostanie wyliczony czas
trwania rezerwacji w dniach jako różnica między \texttt{checkout} a
\texttt{checkin}. Stworzymy nową kolumnę \texttt{lead\_time\_days},
która będzie reprezentować liczbę dni między datą dokonania rezerwacji a
datą zameldowania (różnica między \texttt{checkin} a datą wyciągniętą z
\texttt{timestamp}). Na podstawie czasu trwania rezerwacji zdefiniujemy
zmienną docelową \texttt{long\_stay}, która przyjmie wartość 1, jeśli
czas trwania rezerwacji wyniesie co najmniej x dni, w przeciwnym razie
przyjmie wartość 0. Docelowo jeśli uda się uzyskać lepsze dane, można
będzie rozważyć użycie atrybutów dających większy sygnał do
przewidywania długich pobytów.

\subsubsection{3.6 Analiza rozkładów
	danych}\label{analiza-rozkux142aduxf3w-danych}

\paragraph{Analiza rozkładu długości
	rezerwacji}\label{analiza-rozkux142adu-dux142ugoux15bci-rezerwacji}
\mbox{}
\nopagebreak

\begin{figure}[H]
	\centering
	\includegraphics[width=0.9\textwidth]{../plots/bookings_nights.png}
	\caption{Rozkład długości rezerwacji w dniach}
	\label{fig:bookings_nights}
\end{figure}
Na podstawie na razie dostępnych danych widzimy że rozkład długości
rezerwacji jest stosunkowo płaski (delikatna przewaga dla wartości
poniżej 7), dla każdej z wartości od 1 do 14. Dlatego proponujemy
ustawić próg x na 7 dni, co pozwoli nam zrównoważyć klasy w zmiennej
docelowej \texttt{long\_stay}. Ostateczna wartość progu może być
dostosowana na podstawie dalszej analizy rozkładu długości rezerwacji i
wymagań biznesowych.

\paragraph{Analiza rozkładu ile dni przed zameldowaniem jest dokonywana
	rezerwacja}\label{analiza-rozkux142adu-ile-dni-przed-zameldowaniem-jest-dokonywana-rezerwacja}
\mbox{}
\nopagebreak

\begin{figure}[H]
	\centering
	\includegraphics[width=0.9\textwidth]{../plots/bookings_lead_time_days.png}
	\caption{Rozkład liczby dni wyprzedzenia rezerwacji}
	\label{fig:lead_time}
\end{figure}
Widzimy, że większość rezerwacji jest dokonywana na krótko przed datą
zameldowania, widzimy wyraźną dominację w przedziale do 30 dni przed
zameldowaniem. Jednak istnieje również zauważalna liczba rezerwacji
dokonywanych z większym wyprzedzeniem, sięgającym nawet kilkuset dni.

\paragraph{Analiza rozkładu zmiennej docelowej
	long\_stay}\label{analiza-rozkux142adu-zmiennej-docelowej-long_stay}
\mbox{}
\nopagebreak

\begin{figure}[H]
	\centering
	\includegraphics[width=0.5\textwidth]{../plots/bookings_long_stay_pie.png}
	\caption{Procent rezerwacji długoterminowych (zmienna \texttt{long\_stay})}
	\label{fig:long_stay_pie}
\end{figure}
Widzimy, że około 52.9\% rezerwacji to rezerwacje długoterminowe (co
najmniej 7 dni), podczas gdy 47.1\% to rezerwacje krótkoterminowe
(poniżej 7 dni). Oznacza to, że klasy w zmiennej docelowej
\texttt{long\_stay} są stosunkowo zrównoważone, co jest korzystne dla
trenowania modeli predykcyjnych.

\paragraph{Analiza rozkładu kwartałów
	zameldowania}\label{analiza-rozkux142adu-kwartaux142uxf3w-zameldowania}
\mbox{}
\nopagebreak

\begin{figure}[H]
	\centering
	\includegraphics[width=1.0\textwidth]{../plots/bookings_checkin_quarter.png}
	\caption{Rozkład kwartałów zameldowania}
	\label{fig:checkin_quarter}
\end{figure}
Widzimy, że pierwsze rezerwacje zaczynają się w drugim kwartale 2011
roku, następnie widzimy dość stabilny wzrost liczby rezerwacji aż do
końca 2019 roku. W 2020 roku widzimy spadek liczby rezerwacji, co jest
zgodne z globalnym trendem związanym z pandemią COVID-19. Po 2020 roku
widzimy ponowny wzrost liczby rezerwacji, co sugeruje powrót do
normalności w branży turystycznej. Wzrost ostro przyspiesza aż do
ostatniego kwartału 2024 roku. Następnie mamy niski słupek w pierwszym
kwartale 2025 roku, co może być związane z brakiem danych za ten okres.

\paragraph{Analiza rozkładu kwartałów z jakich pochodzą
	oferty}\label{analiza-rozkux142adu-kwartaux142uxf3w-z-jakich-pochodzux105-oferty}
\mbox{}
\nopagebreak

\begin{figure}[H]
	\centering
	\includegraphics[width=0.7\textwidth]{../plots/listings_last_scraped.png}
	\caption{Rozkład kwartałów, z jakich pochodzą oferty}
	\label{fig:listings_scraped}
\end{figure}
Widzimy, że oferty w naszym zestawie danych pochodzą tylko z pierwszego
kwartału 2025 roku. Jest to duży problem, bo nie mamy wiedzy o tym, jak
oferty wyglądały w przeszłości. Standard obiektów mógł się zmieniać w
czasie, podobnie jak ceny i dostępność. Cechy z \texttt{listings.csv} to
snapshot z 2025Q1, więc mogą nie odzwierciedlać stanu historycznego
ofert. To w zasadzie wyklucza użycie cech z \texttt{listings.csv} do
trenowania modelu predykcyjnego dla naszego problemu.

\paragraph{Analiza rozkładu minimalnej liczby nocy
	rezerwacji}\label{analiza-rozkux142adu-minimalnej-liczby-nocy-rezerwacji}
\mbox{}
\nopagebreak

\begin{figure}[H]
	\centering
	\includegraphics[width=0.9\textwidth]{../plots/listings_minimum_nights_capped14.png}
	\caption{Rozkład minimalnej liczby nocy rezerwacji}
	\label{fig:min_nights}
\end{figure}Widzimy, że większość ofert ma minimalną liczbę nocy ustawioną na 1, co
oznacza, że można dokonać rezerwacji na jedną noc. Jednak istnieje
również zauważalna liczba ofert z wyższymi minimalnymi wymaganiami,
sięgającymi do 14 nocy i więcej. Potencjalnym problemem jest to, że w
naszych danych nie ma żadnych rezerwacji dłuższych niż 14 nocy, więc te
oferty wogóle nie są reprezentowane w danych rezerwacji.

\subsubsection{3.7 Model baseline}\label{model-baseline}

Stworzyliśmy dwa proste modele bazowe do przewidywania zmiennej
docelowej \texttt{long\_stay}:

\begin{enumerate}
	\def\labelenumi{\arabic{enumi}.}
	\tightlist
	\item
	      Model zwracający zawsze 1 (długoterminowa rezerwacja). Osiągnął on
	      52.9\% dokładności, co jest zgodne z rozkładem klas w danych.
	      Natomiast wartość ROC AUC wyniosła 0.5, co oznacza brak zdolności
	      rozróżniania między klasami.
	\item
	      Model oparty na regresji logistycznej wykorzystujący cechy wyłuskane
	      tylko z pliku \texttt{sessions.csv} i \texttt{users.csv}, bez cech z
	      \texttt{listings.csv} ze względu na problemy z aktualnością danych i
	      złączeniami. Cechy użyte w modelu regresji logistycznej to:

	      \begin{itemize}
		      \tightlist
		      \item
		            \texttt{lead\_time\_days}: liczba dni między datą dokonania
		            rezerwacji a datą zameldowania, wyliczona jako różnica między
		            \texttt{checkin} a datą wyciągniętą z \texttt{time\ stamp}.
		      \item
		            \texttt{checkin\_month}: miesiąc zameldowania wyciągnięty z daty
		            \texttt{checkin}.
		      \item
		            \texttt{checkin\_year}: rok zameldowania wyciągnięty z daty
		            \texttt{checkin}.
		      \item
		            \texttt{checkin\_dow}: dzień tygodnia zameldowania wyciągnięty z
		            daty \texttt{checkin}.
		      \item
		            \texttt{checkin\_is\_weekend}: czy data zameldowania przypada na
		            weekend (sobota lub niedziela).
		      \item
		            \texttt{user\_city}: miasto użytkownika wyciągnięte z pliku
		            \texttt{users.csv}.
		      \item
		            \texttt{booking\_month}: miesiąc dokonania rezerwacji wyciągnięty z
		            daty w \texttt{timestamp}.
		      \item
		            \texttt{booking\_dow}: dzień tygodnia dokonania rezerwacji
		            wyciągnięty z daty w \texttt{timestamp}.
		      \item
		            \texttt{booking\_hour}: godzina dokonania rezerwacji wyciągnięta z
		            daty w \texttt{timestamp}.
		      \item
		            \texttt{lead\_time\_bucket}: kategoryzacja \texttt{lead\_time\_days}
		            na przedziały (1,2-3,4-7,8-14,15-30,31-90,91+).
		      \item
		            \texttt{city\_missing}: czy miasto użytkownika jest brakujące w
		            danych. Model regresji logistycznej osiągnął dokładność około 51.6\%
		            oraz wartość ROC AUC około 0.505 na zbiorze testowym. Wyniki te są
		            nieco lepsze niż model bazowy zwracający zawsze 1, ale nadal
		            wskazują na ograniczoną zdolność rozróżniania między klasami.
	      \end{itemize}
\end{enumerate}

\begin{longtable}[]{@{}
	>{\raggedright\arraybackslash}p{(\linewidth - 10\tabcolsep) * \real{0.1304}}
	>{\raggedleft\arraybackslash}p{(\linewidth - 10\tabcolsep) * \real{0.1739}}
	>{\raggedleft\arraybackslash}p{(\linewidth - 10\tabcolsep) * \real{0.1739}}
	>{\raggedleft\arraybackslash}p{(\linewidth - 10\tabcolsep) * \real{0.1739}}
	>{\raggedleft\arraybackslash}p{(\linewidth - 10\tabcolsep) * \real{0.1739}}
	>{\raggedleft\arraybackslash}p{(\linewidth - 10\tabcolsep) * \real{0.1739}}@{}}
	\toprule\noalign{}
	\begin{minipage}[b]{\linewidth}\raggedright
		feature
	\end{minipage} & \begin{minipage}[b]{\linewidth}\raggedleft
		                 pearson\_corr
	                 \end{minipage} & \begin{minipage}[b]{\linewidth}\raggedleft
		                                  spearman\_corr
	                                  \end{minipage} & \begin{minipage}[b]{\linewidth}\raggedleft
		                                                   mutual\_info
	                                                   \end{minipage} & \begin{minipage}[b]{\linewidth}\raggedleft
		                                                                    cramers\_v
	                                                                    \end{minipage} & \begin{minipage}[b]{\linewidth}\raggedleft
		                                                                                     chi2\_p
	                                                                                     \end{minipage}                                                                                                             \\
	\midrule\noalign{}
	\endhead
	\bottomrule\noalign{}
	\endlastfoot
	user\_city                                  & ---                                        & ---                                        & 0.005817                                   & 0.020729                                   & 9.69e-01 \\
	checkin\_year                               & -0.008249                                  & -0.012210                                  & 0.000397                                   & 0.028183                                   &
	7.66e-08                                                                                                                                                                                                                                   \\
	lead\_time\_bucket                          & ---                                        & ---                                        & 0.000380                                   & 0.027508                                   & 1.07e-09 \\
	booking\_hour                               & 0.001695                                   & 0.001692                                   & 0.000119                                   & 0.015417                                   & 7.90e-01 \\
	checkin\_month                              & -0.002166                                  & -0.002148                                  & 0.000066                                   & 0.011485                                   &
	5.18e-01                                                                                                                                                                                                                                   \\
	checkin\_dow                                & 0.005197                                   & 0.005178                                   & 0.000061                                   & 0.011075                                   & 1.51e-01 \\
	booking\_dow                                & 0.003383                                   & 0.003405                                   & 0.000056                                   & 0.010622                                   & 2.77e-01 \\
	booking\_month                              & -0.001477                                  & -0.001466                                  & 0.000047                                   & 0.009668                                   &
	8.45e-01                                                                                                                                                                                                                                   \\
	checkin\_is\_weekend                        & 0.005203                                   & 0.005203                                   & 0.000014                                   & 0.005174                                   &
	1.51e-01                                                                                                                                                                                                                                   \\
	city\_missing                               & 0.001506                                   & 0.001506                                   & 0.000001                                   & 0.001480                                   & 6.81e-01 \\
	lead\_time\_days                            & -0.006086                                  & -0.000548                                  & 0.000000                                   & ---                                        & ---      \\
\end{longtable}

W powyższej tabeli przedstawiono różne miary statystyczne oceniające
związek między cechami a zmienną docelową \texttt{long\_stay}. Wartości
pearsona i spearmana są bardzo niskie dla wszystkich cech, co wskazuje
na słabą liniową i monotoniczną zależność. Miary informacji wzajemnej
również są bardzo niskie, co sugeruje, że cechy te dostarczają niewiele
informacji o zmiennej docelowej. Wartości Cramér's V i p-wartości z
testu chi-kwadrat również wskazują na brak istotnych zależności między
cechami a \texttt{long\_stay}. Ogólnie rzecz biorąc, wyniki te sugerują,
że dostępne cechy mają ograniczoną zdolność predykcyjną dla naszego
problemu.

\subsubsection{3.8 Podsumowanie}\label{podsumowanie}

Spróbowaliśmy zbudować modele bazowe do przewidywania długoterminowych
rezerwacji na podstawie tych danych które nie były wybrakowane oraz dały
się połączyć na podstawie pary kluczy identyfikujących. Modele te
osiągnęły jedynie nieznacznie lepsze wyniki niż proste modele bazowe, co
wskazuje na ograniczoną zdolność predykcyjną dostępnych cech. Głównym
wyzwaniem jest brak historycznych danych ofert w \texttt{listings.csv},
co uniemożliwia wykorzystanie potencjalnie istotnych cech ofert do
trenowania modeli. Również problemem jest brak możliwości złączenia
wszystkich rezerwacji z cechami ofert ze względu na niezgadzające się
identyfikatory. Aby poprawić wyniki, konieczne może być pozyskanie
bardziej kompletnych danych historycznych ofert oraz lepsze złączenie
danych między plikami. Przydatne mogłyby się też okazać dane dotyczące
konkretnych rezerwacji, skąd moglibyśmy pozyskać ceny rezerwacji i inne
cechy.

\end{document}
